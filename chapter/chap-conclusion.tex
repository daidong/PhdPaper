\chapter{结束语}

\section{研究工作总结}

本人博士期间工作和创新主要分为三部分:
\begin{enumerate}

\item 首先对现有的云计算基础软件架构的一个重要实现—Hadoop进行了系统的研究和应用测试,通过提出基于模糊逻辑的异构Hadoop集群的配置工具,简化了大规模异构的Hadoop集群的配置方式,提升了Hadoop集群上应用的执行速度。实验证明通过对Hadoop大规模集群的配置参数进行模糊推理和自动配置能够很大的提高系统性能。但是受限与MapReduce模型以及HDFS存储系统本身的特点,当面对基于海量数据的实时应用的时候,当前的云计算基础软件架构依然面临较大的挑战。应对这些挑战成为博士期间的核心工作。

\item 针对实时应用对分布式存储系统的数据访问速度的要求,我们设计实现了一个完全基于内存的分布式键值存储系统Sedna。通过使用内存作为主要存储介质,极大地提高了随机读写的响应速度和读写带宽。Sedna利用多备份实现了数据持久保存的能力,通过提出一种层次化的数据中心内存储系统的架构设计,Sedna结合一致性哈希和元数据集中管理的优点,提供了更好的扩放性性以及更灵活的负载均衡策略。我们还首次在Sedna中引入了基于触发器的文件读原语,通过监控数据的改变来帮助应用程序实时检测到数据的改变。

\item 针对新型的云计算平台上的复杂应用,我们提出了一个基于触发器的通用编程模型Domino,它能够更好的支持复杂的计算过程,比如典型的多次迭代和递增计算过程。除了将触发器模型引入到分布式处理中,并扩展成为一个通用的编程模型,Domino提出了聚合模式来解决多子任务数据同步的问题。Domino提出了多版本数据管理策略来实现最终同步的方法,进一步解决了在触发器模型下的数据同步问题,极大的扩宽了该模型的应用范围。我们还在Domino系统中首次完整的提出了对计算和数据进行容错以及恢复的算法。最后通过在不同应用上的使用实例也证明了Domino的可行性和效率。

\end{enumerate}


\section{对未来工作的展望}

通过之前的工作,我们建立了一个结合了Sedna和Domino的云计算环境的机会。通过扩展Sedna系统使其支持多版本数据的存储和管理,我们可以轻易的将现基于HBase的Domino编程模型移植到Sedna中。由于Sedna是一个完全基于内存的存储系统,通过和HBase的简单性能对比也可以看出来,其性能远远超过HBase,我们相信和基于Sedna的分布式存储系统的结合将进一步的提升Domino的计算速度。

其次Domino模型本身作为一种触发器模型更加适合递增计算,通过我们的之前的描述也容易注意到,对于某些复杂的大数据的数学计算其并不是非常适合,需要进行问题的转换后才能在Domino模型下实现。因此我们希望能够将Domino模型和现有的基于MapReduce的模型进行结合。对于存在大量数据时的首轮计算将采用MapReduce的方式计算,之后的改变则使用Domino模型进行计算。

另外,对于一个部署了Domino和Sedna的大规模集群来说,一样存在着资源管理优化配置的问题。我们相信基于模糊逻辑的Hadoop异构集群中提出的模糊逻辑思路对于提高集群的效率能够起到很好的指导作用。通过设计Domino和Sedna使其对动态改变配置更加友好,我们相信未来能够设计和实现一个运行时动态配置大规模集群的工具,用于进一步优化集群性能。



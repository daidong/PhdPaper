\chapter{绪论}
\label{chap:introduction}

\section{云计算的基本概念}

进入新世纪以来,计算机技术特别是半导体技术、存储技术、网络技术等领域上的巨大进步,使得人们有能力构建具有大量计算能力
和存储能力的数据中心或大规模集群。在这些大规模集群的基础上,人们开始
将计算机资源看做是像水资源和电资源一样的公共资源,希望提供一种随需随得、
按需付费、多人共享、可监督可测量的新型使用模型。在2006年的搜索引擎大会上,Google时任CEO施密特
首次完整的提出云计算的概念(这个概念最早成为\textit{云端计算},来源于Google工
程师的Google 101项目),立刻被学术界和工业界所广泛接受,成为描述
当前基于互联网计算模式的标准概念。

云计算的概念本身包括的内涵非常广泛,学术界也一直没有标准的定义。
维基百科将云计算\cite{cloudcomputing}描述为一种基于互联网的计算方式。通过这种方式可以将共享的软硬
件资源和信息按需提供给最终用户。从学术的角度来看,本文认为云计算是一门包括了分布式计
算(distributed computing)、并行计算(parallel computing)、效用计算
(utility computing)、分布式存储(distributed storage)、虚拟化
(virtualization)以及数据中心网络(data center)等学科在内的,关联了计算
机系统软件、数据库和网络的综合学科。

按照服务模式划分,云计算可以被分为三个层次:IaaS、PaaS、以及SaaS,按照
服务的对象可以划分为公有云,私有云以及混合云。这里我们按照服务模式的
区别来划分云计算的不同类型。
\begin{enumerate}
\item IaaS(Infrastructure as a Service,即基础架构即服务),主要关注虚
  拟化技术。通过对硬件资源(包括CPU、I/O、网络等)的虚拟化,为用户提供更
  为灵活,低成本的基础架构服务。因此,虽然IaaS主要依靠软件技术来实现,
  但本文中我们仍称之为云计算的基础硬件架构。其中典型的研究内容如各种虚
  拟技术、资源调度算法、虚拟机节能等;而典型的商业应用如Amazon的
  $EC^2$\cite{amazonec2}(Elastic Computing Cloud),
  RackSpace\cite{rackspaceiaas}等。用户在这种服务中不仅仅可以按需获得
  计算和存储资源,还可以控制防火墙、网络拓扑等,然而所有这些操作都是在虚
  拟化的基础之上完成的,用户无法接触到真正的物理设备。这也是IaaS与传统的
  ISP托管服务的标志性区别。

\item PaaS(Platform as a Service,即平台即服务),主要关注并行和分布式
  技术,通过将大规模分布式集群抽象成为统一的存储和计算服务来为用户提供
  分布式系统平台。用户在这种服务模式中获得的是对应用程序运行环境的控制,
  用户无法接触到实际的物理机器以及虚拟机器,所有的资源都通过API抽象给
  用户程序使用,比较典型的例子如Googe App Engine\cite{googleappengine},或者Amazon DynamoDB\cite{dynamoDB}等服务。由于效率和速度的因素,提供此类服务时多直接使用物理机,通过在
  跨域部署的数据中心上构建一层分布式的软件系统来提供这种稳定的、性能可
  扩放的平台服务。我们称这类软件系统为\textbf{云计算的基础软件架构},
  这也正是本文研究的课题。

\item SaaS(Software as a Service,即软件即服务),主要为用户提供各式各
  样的基于云计算基础软硬件平台的软件服务,这些软件不同于传统的单机部署
  的私有软件,在云计算技术的支撑下,这些软件具备了更好的随需扩放,随处
  访问等能力,比如Saleforce公司提供的在线CRM套件等。在这种服务模式下,
  用户直接使用服务提供商提供的应用程序,并不会接触到物理机器、虚拟机、
  操作系统、数据库访问API等层的数据。

\end{enumerate}

云计算一经提出便发展迅速,大量的企业利用PaaS以及IaaS提供的服务为最终用
户提供SaaS服务,这也被证明是一种非常有效的商业模式。不过相比之下,在
IaaS层和PaaS层上,不仅仅出现了新的按需服务的商业模式,更重要的是出现了
许多新的技术和思想,不断提高着人们对于计算机技术本身的理解和掌握。特别
在PaaS领域,由于需要通过管理大量的服务器为用户提供统一的存储和计算能力,
这都对底层操作系统、文件系统、存储系统、以及计算模型等设计都提出了新的
挑战,因此在云计算基础软件方向,新的技术更是层出不穷。

\section{研究背景和意义}

本论文的研究内容是云计算中PaaS层的一个子方向:云计算中分布式存储及分布
式计算模型。这两者也是\textbf{云计算软件体系架构}中的核心组件。

\subsection{分布式存储系统的发展和新的需求}
分布式存储通过管理多台计算机协同提供存储能力。相比较传统存储服务它能够
提供更高的存储容量、更好数据备份和容错能力,也能够提供更高的数据访问速率。按
照其存储数据类型的区别,可以分为分布式文件系统和分布式对象存储系统;按
照应用场景的不同,也可以分为网络文件系统、并行文件系统,以及分布式存储
系统。

网络文件系统(network file system)是将数据维护至分布式环境下的一种解决
方案。它是一种典型的客户机/服务器的系统:数据存储在存储节点上,客户机
通过和访问本地文件系统相同的方式来访问位于存储节点中的数据,通常情况下
存储节点位于网络的另一个位置。比较典型的系统包括Sun公司的NFS系统\cite{Sandberg85designand}以及CMU开发的AdnrewFS系统\cite{howard1988scale}。

由于受到设计思路的限制,这些存储系统往往存储能力有限,面对海量的数据以
及大量的服务器构成的机群情况下很难处理,且客户机/服务器
的模式使得客户机和存储节点之间的网络带宽成为系统最大的瓶颈。在这种情况
下,Google提出了GFS\cite{ghemawat2003google}的解决方案。GFS基于大量的廉价PC以及其上附带的存储设备构建了一个可扩展的文件系
统。与NFS等传统的存储系统不同点在于,GFS在设计之初就考虑了海量服
务器和廉价PC带来的高失败率,通过多备份和对数据写操作进行限制极大的提高
了分布式存储系统的可用性和扩展性。在GFS的基础之上,2006年Google实现了BigTable\cite{chang2008bigtable}。
Bigtable不是一款文件系统而是一个基于GFS实现的分布式对象存储系统,其中所存
储的数据是键值对数据。Bigtable提供了与传统数据库类似的表结构,通过降低
对数据读写的ACID要求提供了传统关系数据库所不能提供的良好的扩放性和容错
性。

Hadoop\cite{hadoopproject}是Apache基金会开发的开源GFS的实现。它不仅仅
包括了GFS这一分布式存储系统(在Hadoop中被称为HDFS),也包括了Bigtable的
开源实现(HBase),以及后面会讲到的MapReduce的实现。通过对Google核心系统
的开源实现,Hadoop实现了基于廉价PC设备的大规模存储能力,并且通过在
Yahoo!, Amazon, Facebook, Twitter等公司的大量应用已经成为了业界标准。

GFS以及Hadoop的大规模使用使得人们开始重新考虑云计算对分布
式存储系统的要求和影响,从而产生了新的设计思路:1)抛弃传统
的单一高性能节点或者专用硬件来提高速度的做法,转而考虑如何在大量的廉价
设备上实现高性能的存储能力;2)扩放行和容错性的重要性被前所未有的提高
。扩放性和容错性是衡量一个云存储系统的核心指标;3)基于对象的存储系统越来越重要,因为基于对象的存储系统介于文件系统和数据库之间,能够为数据提供结构化的语义。 

因此,从2006年开始,大量的分布式存储系统被应用于生产系统中。Dynamo\cite{decandia2007dynamo}是Amazon公司设计实现的一种高可用的键值存储系统,它通过高可用的特性保证了写操作始终成功。与其说其是一款分布式存储系统不如说它事实上定义了分布式存储系统的分布式管理架构,而真正的数据存储则可以简单的由各式各样的本地文件系统实现并且接入。Dynamo的设计中引入了点对点网络的概念,使得集群的扩放性极佳。相比较GFS,Dynamo提供了
基于键值对的数据模型,更加适合存储小数据。类似的系统还有
Cassandra\cite{lakshman2010cassandra},其结合了Dynamo和Bigtable,为应
用提供了结构化的数据语义。

类似Dynamo的点对点式的分布式存储系统很好的解决了存储系统的扩放性和容错性,但是却没有能够为上层应用提供足够快速的数据访问能力。RamCloud\cite{ousterhout2010case}是斯坦福大
学提出的一种完全基于内存的分布式存储系统,首次提出了在云计算环境下完全
使用内存构建持久的分布式存储的可行性和基本技术,并且对于读写性能的提高
给出了非常乐观的估计。RamCloud主要解决的是数
据备份和恢复在基于内存的系统中如何实现的问题,在其后续文章
\cite{ongaro2011fast}中,作者提出了异步备份,并行恢复的策略,被认
为能够有效的解决内存存储系统的持久性问题。

分布式存储系统作为云计算基础软件架构的核心组件之一,在云计算时代到来之初面临了一系列重大的改革,然而这种变革依然在继续。我们知道传统的分布式存储系统把数据存储在网络中不同服务器的硬盘中,
这样当用户发起一个请求时首先将请求分拆,通过网络并发请求给不同的服务器,服务
器从本地硬盘中检索并读取数据,最后通过网络返回到请求节点中。根据这个流程,图
\ref{table:queryspeed}粗略估计了当前工艺下普通PC磁盘以及不同网络条件下,
一次请求的时间估计,从中可以容易的看到磁盘寻道(平均10ms)在请求的数据量较小(1Kb)的情
况下所花费的时间远远超过了网络传输和磁盘数据传输的时间;随着传输的连续
数据量逐渐变大,寻道时间占据的比重越来越低。这也就意味着,当前分布式存
储系统的速度极限受到磁盘本身性能的限制,也受到了数据访问模式的限制。当
主要的数据访问是随机的小数据访问时,保证系统性能就成了不可能的事情。然而在实际应用中这种对小数据的海量存储访问的模式却普遍存在着。比如大量传感器构成的传感器网络,它们不断搜集着数据并且需要持久存储。在这种情况下,一个能够存储海量小数据,并且支持随机访问的云平台下数据访问将极大的提高云计算平台的应用范围。

\begin{table}[h!]\small
  \caption{数据中心请求各个步骤的时间估计}
  \label{table:queryspeed}
  \centering
  \begin{tabular}{|c|c|c|c|c|}
    \hline
    \textit{请求的连续数据} & \textit{网络传输时间} & \textit{寻道时间}
    & \textit{磁盘读取时间} & \textit{总时间} \\
    & \textit{(1Gb/10Gb)} & & &\\
    \hline
    1Kb & $10^{-6}$/$10^{-7}$s & $10^{-2}$s & $10^{-5}$s & $\approx 10^{-2}$s\\
    \hline
    1Mb & $10^{-3}$/$10^{-4}$s & $10^{-2}$s & $10^{-2}$s & $\approx 2*10^{-2}$s\\
    \hline
    1Gb & $1$/$10^{-1}$s & $10^{-2}$s & $10$s & $\approx 11$s\\
    \hline
  \end{tabular}
\end{table}

\subsection{分布式计算模型的发展和新的需求}
分布式计算模型是云计算软件体系架构的另一个核心组件,它负责为用户提供基本的
编程语义,运行时系统支持,以及任务调度、容错、安全审计等。在云计算之前,
大量的分布式系统是通过使用消息传递的方式来实现的,其中比较常用的如高性
能计算中常见的MPI,OpenMPI等;企业级开发中的RMI(远程过程调用)等。2006
年Google提出了一种通用的用于大规模分布式计算的
MapReduce\cite{dean2008mapreduce}模型。该模型将用户程序抽象成为map和
reduce两个阶段组成的一轮计算组成,多轮计算得到最终结果,运行时系统会自
动将map和reduce拆分到不同的服务器执行,并且支持错误恢复,动态任务调度
等功能。MapReduce模型和GFS一起构成了Google的大规模计算的基石。Hadoop项目中
包括了一个Java实现的MapReduce模型,用来和HDFS一起工作以对大规模数据进
行处理。MapReduce是一种批处理模型,map和reduce都需要较长的时间启动执行,map和
reduce之间还存着这强制的数据同步。根据Google内部的统计,在Google的数据
中心中每天有超过1千个MapReduce在运行,短的需要几十秒,长的可能需要运行
一个星期,而平均的运行时间超过10分钟。

鉴于MapReduce的批处理特性,其不支持对数据的实时处理和持续处理。为了支
持大规模实时应用,来自Yahoo!的数据分析科学家提出了
S4\cite{neumeyer2010s4}流式计算模型。该模型将应用程序 
分拆成基于actor实现的小的计算单元,将数据组织成流通过这些计算单元并且
实时的得到结果。类似思路的系统还包括了来自Twitter工程师提出的Storm系统,
该系统的逻辑架构类似MapReduce,它将计算拆分成\textit{spouts}和\textit{bolts}两部
分,前者消耗流式数据,后者产生新的数据流。通过将数据引入\textit{spouts}来启
动storm的执行。流式计算的一个典型缺陷在于容错的复杂性,由于流式计算任
务常驻,任务之间的关系通过数据流产生和维持,当节点传输时数据丢失或者节
点计算过程中失败,任务和当前计算的状态很难在别的节点上恢复,并且恢复时
间过长的话更会导致系统长时间等待。在实际应用中,Storm这样的实时计算模型
通常和MapReduce一起工作,这样即便Storm的中间结果出现错误,最终也能通过
MapReduce任务返回正确结果。

Spark\cite{zaharia2010spark}是由Berkeley AMP实验室提出的分布式计算模型,
基于Scala语言实现。同样是将用户程序拆分成map和reduce两个阶段,Spark通
过RDD(分布式离散数据集)的概念极大的提高了MapReduce的任务执行速度,其性能在全内存的支持下甚至可
以达到MapReduce任务的100倍。RDD是对输入数据的一个抽
象,用户在构造RDD的时候,Spark自动将相关的数据载入到内存中,这个数据
集类似GFS一样不能修改。所有基于该数据集的map,reduce任务都可以更快的完
成。GraphLab\cite{low2012distributed}则是一个专用于图处理的计算模型,
它将分布式的计算抽象成对图中节点和边的处理。每一个节点包含一个计算过程,
不同计算过程之间的数据传递则沿着节点之间的边移动。通过定义不同层次的读
写锁,GraphLab克服了传统的图处理模型只能描述异步程序的限制,成为了一个
通用的分布式编程模型。当然除了它们之外还有很多计算模型不断出现,比如
Dryad\cite{isard2007dryad}, Piccolo\cite{power2010piccolo},
Pregel\cite{malewicz2010pregel}等。

从云计算编程模型发展的历程来看,最开始简单的MapReduce模型专注于如何将复杂的应用抽象成分布式的基本组件,并且能够在
大量的服务器上容错的运行。然而随着云计算环境中应用种类的增加,MapReduce这样简单的模型对于复杂的实时应用、迭代和递增应用的不适应性越来越明显。比如电子商务网站的推荐系统,它们能够根据用户的历史行为为用户推荐有需求的商品。这个算法本身就包含了迭代直至收敛的计算过程,再加上用户的行为在浏览网站的时候也在不断变化,如果希望更准确的预测用户的需求就需要考虑这些最近发生的行为。这样的问题在MapReduce模型上确实无从求解,如何提出并且实现一个高性能的云平台的计算模型以支持这类应用是本文有关计算模型研究的根本出发点。

\section{本文的研究内容}

我们希望提高云计算平台中基础软件架构的性能以使其能够支持复杂
的实时应用的需求。在分布式应用中,\textit{实时}并非指任务必须在某个时间点完成,而是根据
SLA(Service-Level Agreement)的要求尽快给出结果的软实时。比如用户载入自己的社交网络首页,
服务器应该能在500毫秒时间内返回给用户完整的页面。这里所说的\textit{复杂}不是指算法设计上的复杂性,而是指计算过程的复杂性。在分布式系统中,如果一个计算过程内部没有依
赖关系并且不需要全局信息,这个计算就可以被拆分成子任务并行执行并获得最
大的加速比,可惜的是,在大部分情况下都是不可能的。那么计算过程复杂性的区别就体现
在了依赖关系的复杂性和所需的全局信息的数量上。

\textit{复杂}的应用指的是这样一类
计算过程:在应用计算的过程中,每一个阶段都需要来自全局的信息,并且该全
局信息是不断变化的;与此同时,计算包括多个迭代,迭代之间具有数据依赖关
系。更进一步,如果希望输入数据集在计算的过程中不断改变的场景下得到实时的结果,则更是对现有云计算基础软件架构提出的严峻挑战。

前面我们提到了\textit{迭代计算}和\textit{递增计算}的概念。迭代计算(iterative
computation)指应用对同一个数据集进行多次处理直到收敛(达到某种条件),
这其中包含了对重复的数据集进行重复的计算过程。递增计算(incremental
computation)指应用对不断变化(递增)的数据集进行多次处理直到收敛(达到某
种条件)的过程,这其中包含了变化的数据集和基本重复的计算过程。这两种计
算的模式在非常多的算法中都作为核心过程出现,比如PageRank\cite{page1999pagerank}中的迭代收敛过
程以及典型的MLDM(Machine Learning, Data Mining)算法。

本文的研究内容集中在对云计算基础软件架构特别是分布式存储和计算模型的研究上。
我们首先通过对现有云计算基础平台的优化配置来提高现有系统性能(第\ref{chapter:hadoop}章);在认识到
应用的需求和现有系统瓶颈后,通过对分布式存储和计算组件的重新设计,最终为大规模的实时计算提供完整的软件基
础架构支撑(第\ref{chapter:sedna}, \ref{chapter:domino}章)。我们的主要应用对象为:需要在快速响应的\textit{在线的}、包含多轮迭代甚至递增计算模式\textit{复杂的}的大规模应用程序。其中比较典型的包括:社交网络、数据挖掘算法、机器学习算法、深度神经网络算法等。

本研究的主要贡献可以归纳为以下两部分:
\begin{enumerate}
\item 本文通过对完全基于内存的分布式存储系统Sedna的设计和实现为云计算中实时应用提供了远远超过传统分布式存储系统的数据随机访问速度。在设计和实现这一新型的分布式存储系统时,提出并实现一种基于层次化的数据中心存储系统架构;提出了一种基于动态可调整的分布式哈希方案提高了存储节点的均衡;提出并实现了新的专用于实时应用的数据访问API。

\item 本文首次提出了一种完全基于触发器的云平台下的分布式编程模型Domino。面对触发器模型带来的种种挑战,本文提出了聚合模式以及最终同步的概念使得应用可以在不进行同步等待的前提下完成同步计算;结合触发器模型,我们还设计了一套完整的容错和恢复模块,以较小的资源使用率使得触发器的执行可以实时恢复。

\end{enumerate}


\section{论文结构}
本文的组织结构如下:
\begin{enumerate}
  \item 第二章我们介绍一种通过模糊算法自动的优化异构Hadoop集群的方法。通过对已部署的以后Hadoop集群的配置的自动优化,我们能够在不改变软件架构的基础之上提高其性能。
  
  \item 第三章我们将介绍基于内存的分布式存储系统Sedna。该系统是一个适用于实时云计算平台的键值存储系统。相比较现有的分布式存储系统,Sedna提供更快的随机访问速度,更好的扩放性以及特有的实时数据访问接口。
  
  \item 第四章描述一个通用的基于触发器的分布式编程模型Domino。该模型针对包含了复杂的迭代和递增计算的实时应用。通过提供触发语义,提供了更加简洁有效的编程接口。相比较现有的实时计算模型,Domino不仅仅具有更快的执行速度、更高效的对递增计算的支持,提供了更好的错误容忍和恢复的能力。
  
  \item 最后在第五章对本文进行了总结。
\end{enumerate}    
    
    
